\documentclass[a4paper,11pt]{article}

\usepackage[utf8]{inputenc}
\usepackage{fourier}
\usepackage{amsmath,amssymb}
\usepackage{xstring,ifthen,xcolor}
\usepackage{xspace}
\usepackage{url}

\usepackage{color}
\definecolor{orange}{rgb}{0.75,0.5,0}
\definecolor{magenta}{rgb}{1,0,1}
\definecolor{cyan}{rgb}{0,1,1}
\definecolor{grey}{rgb}{0.25,0.25,0.25}
\newcommand{\outline}[1]{{\color{grey}{\scriptsize #1}}}
\newcommand{\todo}[1]{{\color{red}\textit{\textbf{#1}}}}
\newcommand{\note}[1]{{\color{blue}\textit{\textbf{#1}}}}
\newcommand{\citenote}[1]{{\color{orange}{[\textit{\textbf{#1}}]}}}

\usepackage{tikz}
\usepackage{pgfmath}
\usetikzlibrary{calc,shapes,positioning}

\title{Mini-Mapper motor prototype board testing}
\author{Ian~Ross}

\graphicspath{{figs/}}

\begin{document}

\maketitle

This document reports testing of the motor prototype board:

\begin{enumerate}
  \item{Basic tests of the electronics by driving the PWM inputs on
    the board using the Analog Discovery 2, and using the logic
    analyser and scope functions of the AD2 to monitor the motor
    encoder and torque measurement outputs.}
  \item{Tests of the software running on the STM32F767ZI Nucleo-144
    board, connected to the Analog Discovery 2 to monitor PWM outputs
    and provide synthetic inputs for the motor encoder and torque
    measurement code.}
  \item{``Live'' testing of the software, connecting the motor board
    to the Nucleo-144 board.}
\end{enumerate}

The software used for these tests is the \texttt{shell-motor}
application, which provides a command line interface for basic motor
operations. (Once this is all working, the emphasis will shift to
using a Python test bench GUI on a PC to talk to driver code on the
Nucleo board.)

\section{Motor board initial testing}

\subsection*{Setup}

The motor board and AD2 are set up as follows for this series of
tests:
\begin{itemize}
  \item{Connect 5V wall wart power supply to motor board.}
  \item{Connect 3.3V supply from AD2 to motor board.}
  \item{Configure two AD2 digital outputs for \texttt{MOTOR\_IN1} and
    \texttt{MOTOR\_IN2}.}
  \item{Configure AD2 digital input for \texttt{PULSE}.}
  \item{Configure AD2 analog input for \texttt{MOTOR\_SENSE}.}
\end{itemize}

The initial protocol for testing here was something like this:
\begin{enumerate}
  \item{Remove 5V power from motor board.}
  \item{Disconnect \texttt{MOTOR\_IN1} and \texttt{MOTOR\_IN2} from
    motor board.}
  \item{Configure AD2 digital outputs in ``Patterns'' panel and
    check.}
  \item{Connect \texttt{MOTOR\_IN1} and \texttt{MOTOR\_IN2} to motor
    board.}
  \item{Apply 5V power to motor board.}
  \item{Run AD2 digital output waveform generator.}
  \item{Record results: motor motion, \texttt{PULSE} and
    \texttt{MOTOR\_SENSE} inputs.}
\end{enumerate}

The control states I tested with this setup were:
\begin{center}
  \begin{tabular}{lcc}
    \textbf{State} & \textbf{\texttt{MOTOR\_IN1}} &
    \textbf{\texttt{MOTOR\_IN2}} \\

    Motor halted   & HIGH           & HIGH \\
    Forward 10\%   & HIGH           & PWM: 10\% LOW \\
    Forward 50\%   & HIGH           & PWM: 50\% LOW \\
    Forward 100\%  & HIGH           & LOW \\
    Backward 10\%  & PWM: 10\% LOW  & HIGH \\
    Backward 50\%  & PWM: 50\% LOW  & HIGH \\
    Backward 100\% & LOW            & HIGH
  \end{tabular}
\end{center}

\subsection*{Results}

\begin{itemize}
  \item{Quite shockingly, everything worked first time!}
  \item{After the initial setup, I didn't bother too much with the
    protocol above, just switching the PWM duty cycle in the AD2
    ``Patterns'' panel. (Which is a more realistic test than I was
    originally planning to do anyway.)}
  \item{The results were pretty much perfect, as far as I can tell.
    There are only two things that might need some attention.}
  \item{Pulse times from the motor encoder might need some smoothing,
    because the encoder disk isn't quite straight.}
  \item{The signal from the motor coil current sense amplifier that
    I'm going to use for torque measurements is quite noisy, but that
    may just be due to seeing the PWM switching frequency coming
    through into the motor coil current.}
  \item{But: PWM speed control works, forwards and backwards; the
    pulse output from the motor encoder works; and torque measurement
    ``works'', though I need to figure out exactly what I'm seeing
    there.}
\end{itemize}

A better solution for the wobble in the motor encoder signal is
probably to correct the mounting of the encoder disk to the motor
shaft. I designed the hole in the encoder disk under-size to ensure a
tight fit, but it's \emph{too} tight because the encoder disks are
manufactured as PCBs and have two layers of copper with no relief
around the mounting hole. That means that it's not possible to push
the encoder disk onto the motor shaft without expanding the hole in
the disk. I did that in a not very careful way for the disk I'm
currently using, but I have two others, and could try to drill an
accurate 2\,mm hole in them to fit exactly the diameter of the motor
shaft (and then glue them in place).

As for what's going on with the motor coil current signal, I need to
investigate that in more detail, but there is at least something
reasonable there, and the DC level of the signal does go up and down
as the motor drive PWM duty cycle is set to higher and lower values.

\section{Software testing}

\subsection*{Setup}

The Nucleo board and AD2 are set up as follows for this series of
tests:
\begin{itemize}
  \item{Connect ground between Nucleo board and AD2.}
  \item{Connect \texttt{PE5} and \texttt{PE6} on Nucleo
    (\texttt{MOTOR\_IN1} and \texttt{MOTOR\_IN2} motor PWM outputs) to
    logic pins 0 and 1 (inputs) on the AD2.}
  \item{Connect \texttt{PB10} on Nucleo (\texttt{PULSE} motor encoder
    input) to logic pin 2 (output) on AD2.}
  \item{Connect \texttt{PA4} on Nucleo (\texttt{MOTOR\_SENSE} analogue
    torque input) to AD2 analogue waveform output W1.}
\end{itemize}

The idea here is for the software on the Nucleo board to drive the PWM
motor driver signals and for these to be observed by the AD2's logic
analyser. The motor encoder input to the Nucleo board will be driven
by the AD2's pattern generator, and the analogue torque input to the
Nucleo board will be driven by the AD2's signal generator.

The AD2 Waveforms software should be configured as follows:
\begin{itemize}
  \item{Logic analyser with inputs 1 and 2 for motor PWM signals from
    Nucleo board.}
  \item{Pattern generator to logic pin 3 for variable frequency square
    wave to drive motor encode input on Nucleo board.}
  \item{Signal generator to output W1 to drive motor torque
    measurement input on Nucleo board.}
\end{itemize}

Test command script:
\begin{verbatim}
# Ensure initial state and command processing OK.

stop


# Test basic motor commanding.

forward 10
stop
reverse 10
stop


# Test torque measurement.

torque on

# Modify DC level of analogue output from AD2 signal generator to test
# different motor coil current levels.

set torque-interval 5000
set torque-interval 500
torque off


# Test motor speed encoder.

encoder on

# Modify frequency of square wave output from AD2 pattern generator to
# test different motor speed measurements.

set encoder-interval 5000
set encoder-interval 500
encoder off
\end{verbatim}

\subsection*{Results}

\begin{itemize}
  \item{\texttt{stop} gives: DIO0 low, DIO1 high. (WRONG: should be
    both high.)}
    \item{\texttt{forward 10} gives: DIO0 high, DIO1 high. (WRONG: should be
    both DIO0 high, DIO1 PWM.)}
    \item{\texttt{reverse 10} gives: DIO0 PWM, DIO1 high. (WRONG: should be
    both high.)}
\end{itemize}

\subsection{PWM shell regression testing}

\begin{center}
  \begin{tabular}{lccc}
    \textbf{Register} & \textbf{Old: works} & \textbf{New: broken} & \textbf{Status} \\
    \textbf{\texttt{RCC}} \\
    \texttt{AHB1ENR} & \texttt{0x30000a} & \texttt{0x30000f} & OK \\
    \texttt{APB1ENR} & \texttt{0x10040400} & \texttt{0x10040402} & OK \\
    \\
    \textbf{\texttt{TIM10}} \\
    \texttt{CR1}  & \texttt{0x81} & \texttt{0x80} & \texttt{CEN} not set! \\
    \texttt{SR}   & \texttt{0x3} & \texttt{0x1} & No interrupt flag \\
    \texttt{CNT}  & \texttt{0x6c3} & \texttt{0x0} & Not running! \\
    \texttt{CCR1} & \texttt{0x2a30} & \texttt{0x1518} & OK
  \end{tabular}
\end{center}

\end{document}
