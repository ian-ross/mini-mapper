\documentclass[a4paper,11pt,article]{memoir}

\usepackage[utf8]{inputenc}
\usepackage{fourier}
\usepackage{amsmath,amssymb}
\usepackage{xstring,ifthen,xcolor}
\usepackage{xspace}
\usepackage{url}

\usepackage{color}
\definecolor{orange}{rgb}{0.75,0.5,0}
\definecolor{magenta}{rgb}{1,0,1}
\definecolor{cyan}{rgb}{0,1,1}
\definecolor{grey}{rgb}{0.25,0.25,0.25}
\newcommand{\outline}[1]{{\color{grey}{\scriptsize #1}}}
\newcommand{\todo}[1]{{\color{red}\textit{\textbf{#1}}}}
\newcommand{\note}[1]{{\color{blue}\textit{\textbf{#1}}}}
\newcommand{\citenote}[1]{{\color{orange}{[\textit{\textbf{#1}}]}}}

\usepackage{tikz}
\usepackage{pgfmath}
\usetikzlibrary{calc,shapes,positioning}

\title{Mini-Mapper:\,Motor prototype board design notes}
\author{Ian~Ross}

\graphicspath{{figs/}}

\begin{document}

\maketitle

These notes describe the motor prototype board for the Mini-Mapper
robot. This is intended as a platform to test motor driver and motor
encoder setup, and to develop motor early control algorithms
(particularly constant-speed PID control and soft start).


\section*{Requirements}

\begin{itemize}
  \item{Power single Dagu DG01D gearmotor (4.5\,V, 250\,mA max.);}
  \item{PWM speed control, bidirectional, coast/brake functionality;}
  \item{Rotation measurement using optical encoder;}
  \item{Interface to STM32F767ZI Nucleo-144 board;}
  \item{Clean mechanical design for interface between motor and motor
    encoder disk.}
\end{itemize}


\section*{Mechanical design}

\begin{itemize}
  \item{Final PCB to be mounted on top chassis plate of robot: do this
    also for prototype motor board, to get placement right.}
  \item{Principal constraints are:
    \begin{itemize}
      \item{Access to mounting holes on top chassis plate
        (there is interference with the motor under some holes);}
      \item{Clearance under PCB:\@{}both for the possibility of
        mounting components on the bottom of the PCB and for space for
        routing wires and to avoid shorts to chassis plate from THT
        components;}
      \item{Alignment of photoencoder with motor encoder disk:
        photointerrupter slot and optical axis need to align with the
        holes in the motor encoder disk, there needs to be no
        interference between the photointerrupter and the motor body,
        and there needs to be little enough interference between the
        photointerrupter and the top chassis plate to make modifying
        the chassis plate easy.}
  \end{itemize}}
  \item{Motor encoder disk needs to be manufacturable by hand, which
    means a simple disk with holes, rather than slots. This leads to a
    strong alignment requirement in the vertical direction between the
    motor encoder disk and the photointerrupter optical axis.}
\end{itemize}


\section*{Component choices}

\paragraph{Motor driver: Toshiba TB67H450FNG}

Chosen to match supply voltage requirements for motors, to have a
standard PWM-capable control interface, internal current regulation
and an easy-to-use package.

\paragraph{Current sense opamp: TI TLV9051}

This is a low-side current sensing application, which imposes some
requirements on the input range of the opamp. The TLV9051 is
specifically intended for this application.

\paragraph{Photointerrupter: Vishay TCST1202}

Phototransistor type photointerrupter, because we're not going to get
a perfect transistion between full occlusion and full non-occlusion of
the photointerrupter's optical aperture, so it will be useful to be
able to set thresholds for occluded/non-occluded switching. The
primary constraints here are mechanical, since most devices like this
have similar electrical characteristics. The selected device has a
0.5\,mm wide optical aperture, which matches up with some simulations
I've done, and it can be mounted without interference with the motor
body using reasonably sized standard hardware (i.e. 5\,mm standoffs
between the PCB and the top chassis plate).

\paragraph{Motor encoder comparator: ???}

The requirements here are simple, so a jellybean comparator is a good
choice. Not an LM339 though! Something more modern.

\todo{Choose comparator.}

\paragraph{4.5\,V regulator: ???}

Need a linear regulator to produce 4.5\,V from a 5\,V input.

\todo{Choose 4.5\,V linear regulator.}

\paragraph{Power connectors}

\todo{Measure barrel plug from 5\,V wall wart to select barrel jack.}

\paragraph{Nucleo board connectors}

0.1'' headers.

\todo{Decide on connector layout: Nucleo, power, motor.}


\section*{Schematic capture}

\subsection*{Motor driver}

\begin{itemize}
  \item{Power decoupling: following schematic in application note.}
  \item{Current sense resistor: motors run from a 4.5\,V supply, and
    want to keep the maximum sense voltage to some small fraction of
    this. For a maximum current of 250\,mA, a sense resistor of
    400\,$\mathrm{m\Omega}$ gives a maximum sense voltage of 100\,mV.}
  \item{Current regulation: the maximum allowed motor coil current is
    set following the datasheet's instructions for setting the
    $V_\mathrm{ref}$ input.}
  \item{Current sensing: this is a simple non-inverting amplifier
    setup converting the 0--100\,mV sense voltage to a suitable range
    for input to the microcontroller's ADC (0--3\,V). Following a
    suggestion in a TI app
    note\footnote{\url{http://www.ti.com/lit/an/slva959a/slva959a.pdf}}
    about layout for motor drivers, the current sensing connections
    are highlighted to be routed as a differential pair (trying the
    net tie trick suggested in the app note to make the routing less
    confusing).}
\end{itemize}

\subsection*{Motor encoder}

\begin{itemize}
  \item{The LED side of this is simple.}
  \item{On the phototransistor side, ...}
  \item{}
  \item{}
\end{itemize}

\subsection*{Nucleo board interface}

Connections:
\begin{itemize}
  \item{3.3\,V power and ground;}
  \item{Two GPIO outputs for motor control (PWM-capable);}
  \item{One ADC input for motor coil current sense;}
  \item{One GPIO input for photoencoder pulses (interrupt-capable).}
\end{itemize}

Use 0.1'' headers on motor board, and connect to Nucleo board using
jumper cables, since the connections we'll want are pretty spread out.

\begin{center}
  \begin{tabular}{lcccc}
    \textbf{Use} & \textbf{PCB} & \textbf{Nucleo} & \textbf{Conn} & \textbf{Pin} \\
    Power  & \texttt{3V3} & \texttt{+3V3} & CN8 & 7\\
    Ground & \texttt{GND} & \texttt{GND} & CN8 & 11 \\
    Motor control & \texttt{IN1} & \texttt{PA6} (\texttt{TIM3\_CH1}) & CN7 & 12 \\
    Motor control & \texttt{IN2} & \texttt{PA7} (\texttt{TIM3\_CH2}) & CN7 & 14 \\
    Motor current sense & \texttt{Vsense} & \texttt{PA3} (\texttt{ADC1\_IN3}) & CN9 & 1 \\
    Encoder pulses & \texttt{PULSE} & \texttt{PC3}? & CN9 & 5 \\
  \end{tabular}
\end{center}

\todo{Label connections with use and Nucleo pin names on schematic and
  silkscreen.}

\subsection*{Power}

Supplies:
\begin{itemize}
  \item{Need 4.5\,V (250\,mA max.) for motor, and 3.3\,V for logic
    (not sure about current requirements there).}
  \item{For prototype board, use 3.3\,V supply from Nucleo board, and
    use external 5\,V wall wart supply plus linear regulator for
    4.5\,V supply.}
\end{itemize}

\todo{Calculate current draw from 3.3\,V supply and compare to 500\,mA
limit for supply from Nucleo board.}


\section*{Layout}

\todo{Mechanical board outline: photoencoder placement, mounting
  holes.}


\section*{Assembly}

\paragraph{Mounting hardware}

5\,mm spacers between PCB and top chassis plate, M2.5 bolts and nuts
to secure PCB.

\paragraph{Power supply connections}

\end{document}
