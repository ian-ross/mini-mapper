\documentclass[a4paper,11pt,article]{memoir}

\usepackage[utf8]{inputenc}
\usepackage{fourier}
\usepackage{amsmath,amssymb}
\usepackage{xstring,ifthen,xcolor}
\usepackage{xspace}
\usepackage{url}

\usepackage{color}
\definecolor{orange}{rgb}{0.75,0.5,0}
\definecolor{magenta}{rgb}{1,0,1}
\definecolor{cyan}{rgb}{0,1,1}
\definecolor{grey}{rgb}{0.25,0.25,0.25}
\newcommand{\outline}[1]{{\color{grey}{\scriptsize #1}}}
\newcommand{\todo}[1]{{\color{red}\textit{\textbf{#1}}}}
\newcommand{\note}[1]{{\color{blue}\textit{\textbf{#1}}}}
\newcommand{\citenote}[1]{{\color{orange}{[\textit{\textbf{#1}}]}}}

\usepackage{tikz}
\usepackage{pgfmath}
\usetikzlibrary{calc,shapes,positioning}

\title{Mini-Mapper:\,Motor prototype board software}
\author{Ian~Ross}

\graphicspath{{figs/}}

\begin{document}

\maketitle

These notes describe software development for the motor prototype
board for the Mini-Mapper robot. This is intended as a platform to
test motor driver and motor encoder setup, and to develop motor early
control algorithms (particularly constant-speed PID control and soft
start).


\section{Requirements}

\begin{itemize}
  \item{Firmware written in C, using only low-level CMSIS hardware
    definitions.}
  \item{Pick a platform to make development and deployment to the
    Nucleo board as simple as possible for experiments.}
  \item{Standardise on simple ASCII protocol for talking to Nucleo
    board over USB serial port from a PC, to make it easy to write
    experiment front-ends using Python or something similar.}
  \item{Microcontroller features: clocks; power; USB serial; GPIOs for
    status LEDs; GPIOs and timers for PWM motor control; ADC for motor
    coil current sensing; GPIO input, interrupt and timer for motor
    encoder pulse detection.}
\end{itemize}

Cover the following topics in this prototyping phase:
\begin{enumerate}
  \item{Platform setup.}
  \item{CMSIS familiarisation.}
  \item{Blinky: platform test, clock setup, GPIO setup.}
  \item{USB serial shell.}
  \item{PWM demo: more GPIO setup, timer setup, USB serial control,
    view output using AD2.}
  \item{First pass experiment GUI: commanding of motor state (on/off,
    forward/backward, PWM duty cycle), monitoring of motor coil
    current and motor encoder pulses; data saving; real-time
    graphing.}
  \item{Encapsulation of \texttt{Motor}, \texttt{MotorEncoder} and
    \texttt{MotorController} abstractions.}
  \item{Constant speed PID control, including independent optical
    speed measurement.}
  \item{Soft start/ramp control.}
\end{enumerate}

This should be enough for the first ``direct control'' milestone.


\subsection{Setup}

\subsubsection{Platform}


\subsection{CMSIS}


\subsubsection{Blinky}


\subsection{USB serial shell}


\subsection{PWM demonstration}


\subsection{Experiment front-end GUI}


\section{Microcontroller setup}

The Nucleo board has an STM32F767ZI processor on it.

\subsection{Motor driver}

\begin{itemize}
  \item{The Toshiba motor driver has an application note where they
    seem to suggest using $f_{\mathrm{PWM}} = 30\,\mathrm{kHz}$.}
  \item{For a 10-bit resolution PWM duty cycle, this gives a clock
    frequency of $1024 \times 30\,\mathrm{kHz} = 30.72\,\mathrm{MHz}$.}
  \item{The clock input on Nucleo board is 8\,MHz by default. To use
    this, the STM32 should be set to run on the HSE clock (High-Speed
    External) with external bypass (\texttt{RCC\_CR:HSEBYP} = 1,
    \texttt{RCC\_CR:HSEON} = 1).}
  \item{The main clock PLL should be set up with
    \texttt{RCC\_PLLCFGR:PLLSRC} = 1 (HSE), \texttt{RCC\_PLLCFGR:PLLM}
    = 4 (VCO input: 2\,MHz), \texttt{RCC\_PLLCFGR:PLLN} = 128 (VCO
    output: 256\,MHz), \texttt{RCC\_PLLCFGR:PLLP} = 4 (PLL output:
    64\,MHz), and enabled with \texttt{RCC\_CR:PLLON} = 1.}
  \item{The APB prescalers need to be set to make the low-speed
    prescaler give a frequency of less than 45\,MHz:
    \texttt{RCC\_CFGR:PPRE1} = 4 (divide by 2 to give 32\,MHz).}
  \item{The system clock needs to be set to use the PLL:
    \texttt{RCC\_CFGR:SW} = 2.}
  \item{Using a 64\,MHz system clock and a 11-bit resolution PWM duty
    cycle yields a PWM frequency of 31.25\,kHz.}
\end{itemize}


\end{document}
